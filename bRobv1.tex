\documentclass{pnastwo}
%% ADDITIONAL OPTIONAL STYLE FILES
\usepackage[dvips]{graphicx}
\usepackage{pnastwoF}
%\usepackage{article}
\usepackage{amssymb,amsfonts,amsmath}
%% OPTIONAL MACRO DEFINITIONS
\def\s{\sigma}

\begin{document}

\title{Understanding network adaptation in Resting-State fMRI: a information theoretic approach }
\author{Jaime Gomez-Ramirez\affil{1}{Okayama University, Okayama,
Japan}}

\maketitle

\begin{article}

\begin{abstract}
Network theory approaches to brain connectivity in Resting-State fMRI have been
mainly focused on the study of topological properties and network motifs that
characterize the network structure. Small world architectures -highly clustered
nodes connected thorough relatively short paths- have been identified in
healthy and functional brain networks. It has been suggested that the
disruption of normal brain function in diseases such as Schizophrenia and
Alzheimer's disease, can be observed and measured in terms of variations in the
network topology. For example, the reshaping of small-worldness into randomness
in functional connectivity networks. This posits new venues to the study of
brain disease prognosis in terms of network-based biomarkers. However, a formal
understanding of the interplay between brain disease and network connectivity
is still missing. Here we build a information based theory of robustness, in
which rather than quantify how global graph properties -efficiency, global cost
etc.- are modified upon node removal, we study how network connectedness is
altered by biased random walks, that is random walks that follow a given
strategy, for example, favor the visit of nodes with high betweenness
centrality. Thus, each strategy -a vector containing nodes weights- produces a
new network in response of an insult. We calculate then the Kullback-Leibier
distance between these networks that represent different adaptive strategies
and the pre-disturbance network. A new approach for network robustness in
Resting-State fMRI and how it compares to traditional vulnerability-based studies is provided.
\end{abstract}

\keywords{brain function|bias | decision making | subjective value | utility
function}


\section{Introduction}
\dropcap{T}
The method that study BOLD fluctuations that spontaneously emerge during
awake rest is called resting state fMRI (R-fMRI), and the networks that are
identified with such a methodology are resting state networks (RSNs). Resting
state networks have been identified in a panoply of imaging techniques
including fMRI \cite{biswal_functional_1995}, PET \cite {raichle_default_2001}
(first identification of the Default Mode Network), near-infrared spectroscopy
(NIRS) \cite{sasai_nirs-fmri_2012}, EEG
\cite{mantini_electrophysiological_2007}, \cite{boersma_network_2011} and fMRI
combined with diffusion-based studies -diffusion tensor imaging (DTI)
\cite{van_den_heuvel_functionally_2009} and high angular resolution diffusion
imaging (HARDI) \cite{lowe_resting_2008}. Data analysis of resting state data
falls into two gropus: seed-based and model-free methods. In the former, a
functional connectivity map of regions of interest or seeds is obtained. The
map represents the correlation between the resting-state time-series of the
different seeds, which have been a priori selected \cite{cordes_mapping_2000},
\cite{fransson_spontaneous_2005}, \cite{buckner_unrest_2007},
\cite{bluhm_resting_2009}. Model-free methods also provide a connectivity map across brain
regions, but here the regions of interest are obtained through statistical
analysis, rather than defined a priori as in the mode-based approach. 
A number of model-free computational tools exist to analyse
resting-state time series, including independent component analysis
\cite{beckmann_investigations_2005} , \cite{de_luca_fmri_2006}, ,
clustering  \cite{thirion_detection_2006},
\cite{van_den_heuvel_normalized_2008} and machine learning techniques such as
support vector machines \cite{lord_changes_2012}, \cite{zeng_identifying_2012}.
For a critique on the use of the ``minimal assumption'' adopted in model-free
approaches such as ICA in neuroimage data analysis, see
\cite{friston_modes_1998}.

Seed-based methods are relatively straight forward to use and conteptually
simple. A disadventage is that the functional connectivity map here obtained is 
strictely dependent on the regions of interest previosly selected, overlooking
the rest of brain areas and therefore falling to notice potentially
relevant functional connection patterns. Model-free methods, on the other hand,
explore functional connectivity at whole-brain scale. Independent component
analysis allows direct comparison between subject groups
\cite{chen_group_2008}, but presents the disadventage that it provides a number
of different networks (components) whose biological relevance and consistency
with previous findings need to be validated by other means. Note that
model-based approaches avoid this dificulty by adopting the seeds or regions of
interest a priori, either by selecting the relevant regions from a separate
fMRI in a task experiment, or using a brain atlas in rest activity studies
\cite{wang_parcellation-dependent_2009}, \cite{faria_atlas-based_2012}.


Clustering and machine learning methods e.g., SVM, allow individual-based
classification analysis. This is an important advantage to traditional group
analyses of variance, and may bring important insights into disease onset
prediction on individual subject basis \cite{wang_high-dimensional_2010}.
Importantly, the above described approaches taken together show consistency in
their results, that is, the functional maps tend to overlap across
resting-state studies in the human brain \cite{van_den_heuvel_exploring_2010}.


It is important to realize that resting state networks are not
the same as intrinsic connectivity networks (ICNs), which refer to network or components
identified through multivariate decomposition statistical analysis e.g.,
independent component analysis (ICA), that show synchronous
fluctuations during task performance. Therefore, RSNs are a sub
class of ICNs which comprehend a set of large-scale functionally connected brain
networks not only in resting state but also in task-based neuroimaging data.

It has been suggested that fluctuations in the BOLD signal measured
in humans in resting state, represent the neuronal activity baseline and shape
spatially consistent patterns \cite{Raichle:2005}, \cite{Fransson:2006}.
Functional correlation based on the synchrony of low-frequency blood flow
fluctuations in resting state, have been identified in the sensorimotor
\cite{kokkonen_preoperative_2009}, visual \cite{damoiseaux_consistent_2006},
language \cite{hampson_detection_2002}, auditory
\cite{hunter_neural_2006}, dorsal and ventral attention
\cite{fox_spontaneous_2006} and the frontoparietal control system
\cite{vincent_evidence_2008}.

%%%% Review on NDHyp Robustness  on
% R-fmri to give insight on the mechanisms that mediate in
% the NDH


The systematic study of those patterns using correlation
analysis techniques has identified a number of resting state networks, which
are functionally relevant networks found in subjects in the absence of either
goal directed-task or external stimuli. 
The visual identification of the
overall connectivity patters in resting state functional magnetic resonance
imaging (R-fMRI), has been assessed using either model-based and model-free
approaches. In the former, statistical parametric maps of brain activation are
built upon voxel-wise analysis location. This approach has been successful in
the identification of motor networks, but it shows important limitations when
the seed voxel cannot be easily identified. For example, in brain areas with
unclear boundaries; i.e. cognitive networks involved for instance, in language
or memory. Independent Component Analysis (ICA), on the other hand, is a
model-free approach that allows separating resting fluctuations from other
signal variations, resulting on a collection of spatial maps, one for each
independent component, that represent functionally relevant networks in the
brain. While ICA has the advantage over model-free methods that it is unbiased,
(that is, it does not need to posit a specific temporal model of correlation
between ROIs), the functional relevance of the different components is,
however, computed relative to their resemblance to a number of networks based
on criteria that are not easily formalized. More recently, researchers using
graph-theory based methods have been able to not only visualize brain networks,
but to quantify their topological properties. Large-scale anatomical
connectivity analysis in the mammalian brain, shows that brain topology is
neither random nor regular. Instead, it is organized in small world
architectures \cite{Watts:1998}, \cite{Vaessen:2010}, characterized by high
clustering and short path lengths. Small world networks are not solely
structural, functional networks with a small world organization have been
identified in the mammal brain \cite{Bassett:2006}. In addition to
this, disruptions in the small world organization can give clues about normal
development and pathological conditions. For example, Supekar and colleagues
\cite{Supekar:2008} have shown that the deterioration of small world
properties such as the lowering of the cluster coefficient, affect local
network connectivity, which in turn may work as a network biomarker for
Alzheimer's disease. Abnormalities in small-worldness may also have a
significant positive correlation in for example, schizophrenia
\cite{liu_disrupted_2008} and epilepsy \cite{liao_altered_2010}  (Y. Li More
refs here). While network-based
studies have been successful in delineating generic network properties, such as
path length or clustering, additional work is needed in order to come to grips
with the internal working of the systems underlying the network. Robustness in
brain connectivity has been typically approached in terms of the impact that
the complete disruption and/or removal of a network component has in the
network topology \cite{Kaiser:2007}. However, by focusing on the topology
of the network, factors that may play a key role in the network's vulnerability to
failures can be neglected. For example, it has been suggested that patients
with Alzheimer's disease show an increment in brain activity in certain areas
relative to healthy subjects that compensates for the disease related atrophy
of other regions \cite{Sanz-Arigita:2010}. In this paper we explore the network
degeneration hypothesis \cite{Seeley:2009}, \cite{Mesulam:2009}, which states
that neurological diseases target functional neural networks modifying its topological properties, using a
methodology that combining graph theoretic tools and information theory may
rigorously address robustness and its interplay in aging and pathological
conditions. This approach posits a new theoretical framework to investigate
network  robustness and how it is affected by ``internal perturbations" such as
aging and neurological disorders.

Despite the variability in the data acquisition protocols,
statistical data analysis and gropus of subjects employed by different
researchers, the litterature consistenly shows an overlap of functionally linked
networks in the brain during resting state. The most commonly reported resting
state networks are at least eight:  the primary sensorimotor network, the primary visual and extra-striate visual
network, bilateral temporal/insular and anterior cingulate cortex regions, left and right lateralized networks
consisting of superior parietal and superior frontal regions, and the 
default mode network \cite{van_den_heuvel_exploring_2010}.
 



\subsection{Graph Theory:a brain-scale approach}
Until the recent advent of graph theoretic methods in RS-fMRI, the focus was
put on the identification of anatomically separated regions that show a
high level of functional correlation during rest. Grapth theory provides a
theoretical framework to investigate the overall architecture of the brain.
Thus, the emphasis has shifted from the identification of local subnetworks
-default mode network, primary sensory motor network etc.- to the assesment of
the topological and informational characteristics of a brain-scale complex
network. The tools we use to model a system may also convey an ontological
version of it, that is to say, the system under study is seen through the
lens of a specific approach that necessarly shapes the observability domain.
Thus, the identification of different subnetworks during rest can be seen as a
 by-product of the techniques used, for example identification component
 analysis (ICA) or clustering. On the other hand, the emphasis on the overall
 brain functional architecture might not suprised when using graph
 theoretic methods. Notable proponents of a modularist vision of brain
 connectivity to understand cognition -\cite{gazzaniga_new_1999}, \cite{Fuster:2000}- have
 shifted toward a network-based approach- \cite{bassett_understanding_2011}.


\subsection {Clinical applications of RS-fMRI}
 Functional connnectivity studies of neurological and neuropsychitatric
 disorders are being copiously produced. Resting-state functional magnetic
 resonance imaging experiments are considerably less demanding in terms of
 subject oparticipation than performace studies.This is of the outmost
 importance in dealing with subjects (patients) with their cognitive or motor
 capabilities reduced due pathological conditions.
 Altered resting-state functional connectivity patterns have been shown in an
 impressive range of pathologies and conditions - Alzheimer's disease,
 schizophrenia, multiple sclerosis, Parkinson's disease, depression,
 autism, and even attention deficit/ hyperactivity disorder. See
 \cite{lee_resting-state_2012}�for a review on clinical applications of rs-fMRI at the single subject level.
 
 RS- fMRI have been successfully used in classification of subjects with AD
 and MCI versus healthy controls. Connectivity changes in the default mode
 network can be used as markers of pathological conditions. For example,
 patients with tivity in the salience network. Alzheimer disease demonstrated
 decreased connectivity in the default mode network but increased connectivity
 in the salience network.
 Other studies rather than focusing on specific RSNs -default mode network-
 investigate altered connectivity patterns at a brain-wide level. The
 hypothesis that neurological disorders target large-scale
 functional and structural networks \cite{Seeley:2009} rather than specific
 loci or sub-networks, calls for an integrative network-based approach.
 
 Schizophrenia patients show a decreased functional connectivity during rest
 \cite{liang_widespread_2006} suggesting a suboptimal information integration
 between regions of the brain network \cite{micheloyannis_small-world_2006}
 AD patients show clustering coefficients significantly lower in patients compared with
controls. 



\subsection{Network Degeneration Hypothesis in ICN(state of the art)}
 

R-fMRI studies avoids the problem of task-evoked BOLD
fluctuations and is easier to implement in subjects with their cognitive or
motor capabilities reduced as in Alzheimer's or Parkinsons's disease.

The network degeneration hypothesis -disease starts in small
network assemblies, to progressively spread to connected areas of the initial
locus- has been investigated in a number of brain pathologies including
Alzheimer's disease \cite{Buckner:2009}, epilepsy
\cite{liao_altered_2010}, schizophrenia \cite{liu_disrupted_2008} and unipolar
depression \cite{lord_changes_2012}.
%http://www.alzforum.org/new/detail.asp?id=2106
The NDH encompasses the idea that
neurodegeneration can be studied as a network dysfunction process, in which
changes in the network organization are informative about the progression of
the disease. This paves the way for a network-based approach to diagnosis of neural disorders, and the
discovery of network biomarkers in early disease detection. To our knowledge,
the first attempt to systematically test the NDH is in \cite{Seeley:2009},
in which Seeley and colleagues use functional and structural network mapping 
approaches to characterize five distinct neurodegenerative syndromes. 
There is a growing body of evidence that neurodegenerative disease targets
specific large-scale brain networks, and clinical applications  




However, it might be emphasized that the  
empirical validation of the network degeneration hypothesis does not tell us
much about the mechanisms that mediate in the alleged network connectivity
sensitivity to neuropathological syndromes. 
A critical aspect is to understand network robustness, that is, functional
network invariance under perturbation. In essence, robustness measures the
capacity of the network to perform the same function before and after a
perturbation. Perturbations are events, internal or external, that elicit a
change in the network configuration, as for example in, to obliterate a node or
a change in the connectivity between nodes. 
Thus, for a given network  $G(V,E)$ with an adjacency matrix $T$, a perturbation
$d_{A}$ that transforms $T$ into a new adjacency matrix $T_{A}$ is given by the
stochastic map $d_A: T \to T_A$. Here, we assume that the perturbation $d_A$
refers to the set of nodes that remained after having deleted the set of nodes
$V-A$. Therefore, originally the network $G(V,E)$ is transformed into
$G(A,E^{A})$, where $E^{A}$ are the remaining edge that results of eliminate the
nodes $V-A$ from $V$. Contrary to standard robustness studies reviewed in section 
\ref{} that measure the effect of network insults -typically node removal-  
in terms of macroscopic properties such as cost, efficiency or entropy, here we
focus on the different adaptive processes that may follow the network
disruption, and we are able to identify the strategies that promote function
invariance. The Kullback-Leibler distance allows to answer the question of which
of a set of approximating models is closest to the original model $f$. This
is a classical problem of model selection in which the Kullback-Leibler distance
allows elucidate which model in a set of candidate models, $g(x|\theta)$, is the
closest to the true model $f(x)$. Accordingly, the best candidate is that
with minimum distance. Thus, the minimum distance between $f$ and $g$ in the
case of discrete distributions is given by:

\begin{equation}
KL(f,g)= \sum_{i=1} ^{n} p_i \log(\frac{p_i}{q_i})
\end{equation}
where $n$ is the number of possible outcomes of the underlying random
variable, $p_i$ is the probability distribution of the ith outcome and $q_i$ is
the approximating probability distribution.
The KL distance is not a true metric since is not symmetric, as the distance is
different from the distance. For a more in detailed description of KL, see
\cite{Cover:1991}

It is important to note that KL cannot be computed unless we have a full
knowledge of both the true model $f$ and the parameters $\theta$ in $g_\theta$.
While this requirement is often unrealistic for observational studies,
specially in biology, it holds in the present case \cite{Burnham:2010}. The
applicability of this approach relies upon the fact that ``true model'' $f$ is given by the network
adjacency matrix $T$ prior to the insult $d_A$ and the parameters $\theta$ in
$g_\theta$ can be assigned to a vector $b_i={b_{i1},\ldots,b_{i |A|}}$ that
assigns specific weights to each node in $T_A$, the resulting adjacency matrix
of the perturbation $d_A$ readjusted with a bias or strategy $b_i$, in which the
passage through certain nodes are favored related to others. For example, a
bias or adaptive strategy in which random walkers are more likely to visit nodes
with highest betweenness centrality can be constructed straight forward by
increasing the weight of the edges that lead to those nodes.






\subsection{Network Robustness}

network topology that promotes Functional invariance
Figure

\subsection{Model selection}
which probability distribution
\section{Methods}
\section{Results}
\section{Discussion}


\begin{acknowledgments}
This work was partially supported by 

\end{acknowledgments}

\bibliographystyle{abbrv}
%\bibliography{~/Eclipse/workspace/BiblioTex/bibliojgr}
\bibliography{bibliojgr}
\end{article}
\end{document}