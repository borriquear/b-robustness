

1. 
Ask:
- mean age of elder.
- Gender? or all males? Note that "Arecentstudy
by Weissman-Fogeletal.(2010) foundnosignificantdifferences
betweengendersinrestingfunctionalconnectivityofthebrain
areaswithintheexecutivecontrol,salient,andthedefaultmode
networks." http://www.ncbi.nlm.nih.gov/pmc/articles/PMC3635030/pdf/fncom-07-00038.pdf

2.
-Asymmetry hypothesis, network eff. loss in left - eff los in right (for some networks) difference in young should be larger than in old, because in old brain becomes more symmetric to compensate. so difference in lateral losses become less pronounced with age.

3.
-Check if any if these nodes are also those with highest efficiency loss: 
^^^^^^^^^^^I dont see that in my data.^^^^^^^^

"subjects in small adult age ranges.
The SVM method allows for detection of the most influential features and nodes which drive the classifier or predictor. We utilized this approach to find the ?connectivity hubs,? or nodes with the most significant features that influenced age classification. Tables ?Tables33 and ?and55 reveal the 10 most influential nodes for the linear age SVM classifier and for the linear SVR predictor, respectively. Four out of the 10 most influential nodes are present in both methods: R_precuneus_1, R_sup_frontal, L_precuneus_1, and L_sup_frontal (see Figures ?Figures1212 and ?and13).13). There is a similar degree of agreement between the RBF SVR nodes and the linear SVR nodes: L_precuneus_1, L_parietal_1, R_parietal_3, and L_IPL_1 are in both methods. This agreement between classifier and predictor methods suggests that the connectivity of these nodes provides discriminatory information with respect to age differences with some independence of choice of method.
http://www.ncbi.nlm.nih.gov/pubmed/23630491

4.
-  disconnect DMN and HC and check whether "the relationship between the DMN and HC breaks down" this is because in aging , there is  elevated HC at rest which restricts the degree to which HC interacts with other brain regions during memory tasks, and thus results in memory deficits. 
Tbn prueba HC dentro de la DMN:
"Our results show that in healthy adults, hippocampus exhibits reduced functional connectivity across the default mode network, whereas medial prefrontal cortex evidences weaker connectivity with posterior cingulate and parahippocampal regions." https://ww4.aievolution.com/hbm1501/index.cfm?do=abs.viewAbs&abs=4088

-Test if HC is more central in Aging (Salami)
http://www.pnas.org/content/111/49/17654.full
Ojo: en https://ww4.aievolution.com/hbm1501/index.cfm?do=abs.viewAbs&abs=4088    " In contrast to a previous report (Salami), connectivity between the left and right hippocampus was also negatively related to age."



-test : Episodic memories are established and maintained by close interplay between hippocampus and other cortical regions, but degradation of a fronto-striatal network has been suggested to be a driving force of memory decline in aging. "http://cercor.oxfordjournals.org/content/early/2015/05/19/cercor.bhv102.abstract"
It was found that self-esteem is related to the connectivity of frontostriatal circuits, suggesting that feelings of self-worth may emerge from neural systems which integrate information about the self with positive affect and reward.[9]
 ---  dorsal striatum (caudate nucleus and putamen). Stratium is part of the basal ganglia.The main components of the basal ganglia - as defined functionally - are thestriatum (caudate nucleus and putamen), the globus pallidus, the substantia nigra, the nucleus accumbens, and the subthalamic nucleus.

- Justify why in pout case, the efficiency loss in old is much larger tyhan in  young, despite the fact that " decreased activity in older versus younger subjects in 2 resting-state networks (RSNs) resembling the previously described DMN, containing the superior and middle frontal gyrus, posterior cingulate, middle temporal gyrus, and the superior parietal region"  http://www.ncbi.nlm.nih.gov/pubmed/18063564

First check if DMN is the same, if so, It may be the case that in old DMN is less active and then some compenmsatory mechanism in the form of more centrality of the DMN, so when i lesion  the DMN the eff. loss is larger since the DMN is more central , connectivity was arranged in a way to compensate the reduction in DMN activity for old.

This poin t , from Demoisenaux, is also good for my point:  functional connectivity of intrinsic brain activity in the "default-mode" network (DMN) is affected by normal aging and that this relates to cognitive function.

- "The older subjects exhibited significantly lower DMN activity in the posterior cingulate (PCC, t-test P<.001) as well as a tendency to lower activity in all other DMN regions in comparison to the younger subjects. TEST this: We found no significant effect of age on DMN inter-connectivity."
http://www.ncbi.nlm.nih.gov/pubmed/20004726

-For future works: investigate this continuum :"The majority of studies show a decreased DMN functional connectivity and task-induced DMN deactivations along a continuum from normal aging to mild cognitive impairment and to Alzheimer's disease (AD)."

-What can be said about brain metabolism

- DMN highly active is bad for concentration and may enhance depression etc.
"DMN in the healthy brain is associated with stimulus-independent thought and self-reflection and that greater suppression of the DMN is associated with better performance on attention-demanding tasks. In schizophrenia and depression, the DMN is often found to be hyperactivated and hyperconnected. In schizophrenia this may relate to overly intensive self-reference and impairments in attention and working memory. In depression, DMN hyperactivity may be related to negative rumination. ""http://www.ncbi.nlm.nih.gov/pubmed/22224834"
-- 

=========================
Jaime G�mez Ram�rez 
Research Scientist 
University of Wisconsin-Madison
==========================

==========================================================================================
http://www.sciencedirect.com/science/article/pii/S0896627312002279
http://www.sciencedirect.com/science/article/pii/S0896627312001353

http://www.ucsf.edu/news/2012/03/11712/alzheimers-disease-spreads-through-linked-nerve-cells-brain-imaging-studies
   Our next goal is to further develop methods to predict disease progression, using these models to create a template for how disease will progress in the brain of an affected individual

principle appr: The new evidence suggests that different kinds of dementias spread from neuron to neuron in similar ways, even though they act on different brain networks, according to Seeley.

However, both the network-degeneration view and supporting pathological data are descriptive rather than explicative, qualitative rather than model-based.

http://www.sciencedirect.com/science/article/pii/S0896627312001353
http://www.ncbi.nlm.nih.gov/pmc/articles/PMC3777690/
http://researchfund.axa.com/project/professor-harald-hampel
http://www.ncbi.nlm.nih.gov/pubmed/20592748
http://www.alzheimersanddementia.com/article/S1552-5260(13)00832-7/fulltext
http://www.journals.elsevier.com/information-sciences
http://www.sciencedirect.com/science/article/pii/S0020025509002291
http://www.scopus.com/record/display.url?eid=2-s2.0-71849091509&origin=inward&txGid=6094BC2D777FDB63F23642EA8D81D75C.aXczxbyuHHiXgaIW6Ho7g%3a2

http://www.sciencedirect.com/science/article/pii/S1053811911007117


A functional atlas was used to parcellate the brain into spatially segregated units. Functional connectivity networks were derived from inter-regional correlations of spontaneous low-frequency signal fluctuations in the rs-fMRI time series, and graph-theoretical metrics were calculated from the co-activation correlation matrices. In addition, structural connectivity networks were obtained using DTI-based tractography.

http://www.ncbi.nlm.nih.gov/pubmed/18846369
We present and evaluate a new automated method based on support vector machine (SVM) classification of whole-brain anatomical magnetic resonance imaging to discriminate between patients with Alzheimer's disease (AD) and elderly control subjects.
hree-dimensional T1-weighted MR images of each subject were automatically parcellated into regions of interest (ROIs). Based upon the characteristics of gray matter extracted from each ROI, we used an SVM algorithm to classify the subjects and statistical procedures based on bootstrap resampling to ensure the robustness of the results.
RESULTS:
We obtained 94.5% mean correct classification for AD and control subjects (mean specificity, 96.6%; mean sensitivity, 91.5%).
CONCLUSIONS:
Our method has the potential in distinguishing patients with AD from elderly controls and therefore may help in the early diagnosis of AD.

http://link.springer.com/chapter/10.1007%2F978-3-642-02267-8_16
A common feature selection stage is first described, where Principal Component Analysis (PCA) is applied over the data to drastically reduce the dimension of the feature space, followed by the study of neural networks and support vector machines (SVM) classifiers


http://brain.oxfordjournals.org/content/131/3/681.short
Recent advances in statistical learning theory have led to the application of support vector machines to MRI for detection of a variety of disease states
We used linear support vector machines to classify the grey matter segment of T1-weighted MR scans from pathologically proven AD patients and cognitively normal elderly individuals
-- 

=========================
Jaime G�mez Ram�rez 
Research Scientist 
University of Wisconsin-Madison
==========================