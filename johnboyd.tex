% LaTeX file for a 1 page document
\documentclass[10pt]{article}
%\usepackage{e-jc}
\usepackage{amsmath}

\title{On Matching \\ \small{ (Notes from the underground)}} %Conciliating Predictive Coding and Matching}
\author{Jaime Gomez y Ramirez\\ %\thanks{}\\
\small \\[-0.8ex]
\small \\
\small \texttt{ }\\
\and
%Forgotten Second Author\\
%\small School of Hard Knocks\\[-0.8ex]
%\small University of Western Nowhere\\[-0.8ex]
%\small Nowhere Uvherdov\\
%\small \texttt{no1remembers@me.woe.edu}
}

%\date{\dateline{Jan 1, 2009}{Jan 2, 2009}{Jan 3, 2009}\\
%\small Mathematics Subject Classifications: 05C88, 05C89}

\begin{document}
\maketitle

\begin{abstract}
\end{abstract}

%\section{Introduction. John Boyd }
\section{Introduction}
In this paper we investigate the Matching principle. Matching is an operational principle that relates to the functioning of systems interacting in a rich environment, or at least richer than the system itself, in the sense of typical measures e.g., computational complexity.
We argue that matching is an overarching principle that may help to clarify fundamental problems in cognitive and brain science such as the nature of consciousness, perception, development and evolution and decision making.
Rather than positing a new theoretical framework called to compete and ultimately phagocytize alternative approaches with similar epistemological pretensions, for example predictive coding, we aim at identifying the common basis that sustain both approaches, pinpoint the assumptions in which they depart to finally highlight a common ground for both in which Matching and Predicitive Coding.
By comparing the Predicitive Coding approach to the Matching approach, will be shown to shed light new light on applications.
 
\section{Assumptions and motivations}
Predictive coding is a variational principle (free energy minimization) that assumes that biological systems function as homeostatic mechanisms. Matching, on the other hand, is not variational, and assumes that the modus operandi of biological systems is the construction of internal conceptual structures that resonate in effective ways with the external conceptual structure of the world.
Both principles have in common that they predicate on the behavior of systems in its environment, but they fundamentally diverge in how the world is perceived and valued. While in predictive coding the system is incessantly looking for achieving a homeostatic balance with the external world, matching on the other hand, is dedicated to the construction a coherent internal conceptual structure vis a vis the perceived conceptual structure of the outside world. 

To understand the motivation and ultimate results of predictive coding we need to pay attention to the kind of mathematical tools employed, Lagrangian mechanics and Kalman filter. In predictive coding the main goal of a system is to predict the future states of the world, which are necessarily unknown, and need to be inferred, that is, the system is in essence a predictor, the better it predicts the closer it is from equilibrium, which its only one goal. 
It is important to remark that predictive coding works under two assumptions: 
\begin{itemize}
\item The world (signal) is stationary, that is, probabilities are fixed in time
\item The world is ergodic which is a generalization of the law of large numbers (long term averages can be closely approximated by averages  across the probability space)
\end{itemize} 

Matching on the other hand, inherits its conceptual and mathematical apparatus from IIT. The emphasis is here on building a mapping, a correspondence between internal and external conceptual structures. Thus, the system is not maximizing/minimizing any specific quantity, it is rather interested in building up concepts of meaning, that is concepts that relate to the world as it is perceived. It is the creation and destruction of concepts of meaning -concepts that are captured by the Matching mapping (technically a functor)- what permit us to both shape and be shaped by a changing environment. Crucially, the unrealistic assumptions of stationarity and ergodicity are not necessary. Matching can not be avoided if we want to survive in our own terms \cite{}. 
% \emph{"Dialectic changing and expanding universe of concepts matched to a changing and expanding universe of observed reality."} 

%\subsection{Creation and Destruction}
%Destroy and Create mental patterns or concepts of meaning is what permit us to both shape and be shaped by a changing environment. We cannot avoid this kind of activity if we want to survive on our own terms.
%Dialectic changing and expanding universe of concepts matched to a changing and expanding universe of observed reality.

\section{The modus operandi}
Predictive coding claims to be a neurobiologically plausible scheme for inferring the causes of sensory input using Kalman filtering and Bayesian updating that must result in minimizing prediction error.
Under predictive coding the main rationale for the agent's action is to predict the next state of the world, using a Bayesian approach to brain function which is conceived as a Helmholtz machine \cite{Hinton}. A Helmholtz machine is aimed at learning the hidden structure (the external world) by creating a generative model that, it is expected, approximates the data set of the hidden structure.
Thus, the modus operandi of a system implementing this scheme is to minimize prediction error, in doing so it is assumed that the system maintains a homeostatic relation with the world, which is in Friston words, the only thing that matters in physiology. Predictive coding fosters a top-down bottom-up view of perception, internal biases shape perception (top down) and the errors in what it is perceived versus what is predicted are used to correct the bias to make better predictions (bottom-up).  

Matching, on the other hand, is not an inferential machine of any sort, but a mapping between the internal conceptual structures of the system and the external conceptual structure of its world i.e., as it is perceived. Matching does not share the "hierarchical narrative" of predictive coding, it rather, as it may be expected for being a continuation of IIT, it follows a deductive-inductive approach in which axioms and postulates together provide a principled way to characterize (predict and explain) properties of the system in the world.
 
\section{No free lunch}
%%%%%%%%
Superior performance on some functions implies inferior perf. on others (there can be no generally  superior computational function optimizer), so gains and losses balance precisely, so an optimizer has to pay for its optimality on one subset of functions with sub optimality in the complementary subset. (Boden integral th. which is a general. of Data rate th.) 
Information can never be simply gained by processing, because the best algorithm that turns the pb into a solution  will necessarily be highly problem specific. Thus there can be noalg. for all sets of bps although there will be an optimal lag for any given set.
 %%%%%
 
 \section{Deconstructing Matching}
In what follows I try to "deconstruct" Matching with the hope to clarify some of the conceptual difficulties that we are having with the definition and application of Matching. The ideas that are exposed next are still at the gestational-intuitive level and not formalized.

A system matches better if it has many concepts about the world, this is what I call having meaningful with sanity,  that is, having concept that help to predict and explain the world, not multiple concepts for purely endogenous motivations. A system that matches the world wold have not only many concepts but also concept that resonate with what it is "out there". Matching, in this view, can be seen as an activity related to "survival in its own terms". However, it is not survival the only thing that matters,  a system with a large potential to "enjoy life", that is, it has many concepts, values, rich habits etc.will be occupied with enlarging its existence, even if contingently, and not only in extend its lice span.   
Nevertheless, If Matching is indeed a principle of biological systems functionng, it must have a survival value, otherwise it would have been discarded along evolution. To be coherent with IIT we may claim that the instinct for survival implies that a primordial goal is to improve our capacity for cause-effect power, that is, our capacity to exist. Although this may sound a tautology, we survive in order to exist, if we acknowledge the IIT identity, the tautology dissolves, we survive in order to exist and existence is having maximally irreducible cause effect power.
Furthermore, if the identity posited in IIT is correct, ecological statements logically follow. For example, if we believe that we can not alone maximize cause-effect power then we may agree to constraint upon our cause-effect power in order to pool skills and talents in the form of professional associations, mafias, nations etc. so that obstacles that stand in the way of the basic goal (maximize existence or cause effect power) can be removed or overcome \cite{Boyd}.    
Segregation happens, when alienated members of a complex may join another complex (collective body), if in doing so they improve their capacity for existence or cause-effect power. The concepts in a complex can be seen as decision models to improve the capacity for cause effect power.
%Therefore the degree to which we cooperate or compete with others is implicitly established or driven by the need to satisfy this basic goal. 

%survival value
Matching can be stated as the product of Existence and Knowledge or meaningful existence, which can be formalized using IIT as the product of the integrated conceptual information ($\Phi^{Max}$) which measures how much it exists, and how much knowledge it possesses ($\sum \phi$). 

\begin{equation}
M = Existence* Knowledge
\end{equation}

In this view, cockroaches have  large survival value than, for example, rabbits in a poorly vegetated area densely populated with predators, but the Matching of the rabbits will be larger than that of cockroaches.
It may now be argued that Matching is not an universally valid principle for animal behavior. For example, for those that believe that the only goal of existence, if any, is survival and reproduction matching can be seen as lacking evolutionary value. 
It may be, however, pertinent to remind the words of John Stuart Mill "It is better to be a human being dissatisfied than a pig satisfied; better to be Socrates dissatisfied than a fool satisfied. And if the fool, or the pig, is of a different opinion, it is only because they only know their own side of the question.� The crux of the matter is that the cockroach survives but it has low matching "they only know their own side of the question"  but rabbits have higher matching, for example they know that they will not survive long, unless ...
Existence ($\Phi^{Max}$) is tragic \cite{Ortega}, that is, there is always an end, and knowledge is bounded \cit{Simon}(decisions are always limited by  knowledge ($\sum \phi$), these two are the boundaries of Matching. The more one matches the more one exists and knows but this is not a recipe for happiness but a for a larger life, that is, a life in your own terms or according to your own potential.

 \section{How to Match the world?}
 This is a question with multiple answers, so I will content myself try to make clear only one of them that is inspired in IIT but also drinks in other apocryphal waters. 
First, since Matching relies on having as many concepts as possible that are informative about the world, that is, having cause-effect power, we need to understand How to build the concepts?
Here, I will not discuss how to do this in IIT, since this is well exposed in IIT 3.0, but to present unwalked paths that may share the vision and some of the explanans of IIT.

Matching is about creation and destruction. This bombastic point deserves clarification. 
1. The external conceptual structure is calculated by scrambling the world and the differential remaining after this operation is done denotes the structure of the world. Crucially, this operation of shattering or destruction of concepts (by scrambling the world, we destroy the existing destroy concepts plus potential concepts that could otherwise be created) and is related to deduction or analysis. We call to this \emph{destructive deduction}. 
2.  The external conceptual structure needs, a posteriori, to be tested or reconstructed. This is called \emph{constructive induction}. It is worth noting that the in order to have creative induction we need have separated the elements (cuts, scrambling) using destructive deduction. Without the previous destruction, meaning cannot be achieved since the pieces (mechanisms) are tied together without being tested or challenged.  

 The dialectics of unstructuring to then structuring encompasses a way of changing our perception of reality and is in the spirit of Matching. Importantly, it implies that the internal concepts must be \emph{consistent} and \emph{match up} the external reality. Once the match up conceptual structure is built and tested (consistency), it reflects a pattern of ideas -a generative model- that can be used to describe some aspects of the observed reality. Then there is little appeal to modify or expand the concepts in the conceptual structure for consistency, the effort rather goes inward to improve generality and enhance the \emph{matching} of the concept(s) with reality. 
There must be always a mismatch between the internal and the external conceptual structures, that is between the observation (explanandum) and the model of that observation (explanans), otherwise synthesis and analysis would be one and the same. This apparent platitude is, nevertheless, not reckon in Predictive Coding which aims to make deduction \emph{equals} inference or analysis \emph{equals} synthesis. 
   
 \section{Why do we need Matching (so badly)}
 We just mentioned that there must be always a mismatch between the internal and the external conceptual structures, this has a fundamental repercussion. To get to that point we need first to pay visit to two plentifully mentioned but poorly understood theories: G�del's Incompleteness theorems and The second law of  thermodynamics. 
G�del's Incompleteness Theorems are \footnote{1.Any consistent system is incomplete and 2. Even if such a system is consistent its consistency can not be proved within the system, we need to appeal to another system outside the previous one} have something interesting to say about the limitations on the match up of a conceptual structure with the observed reality?
The conceptual structure $CS_W$ is always incomplete since it depends upon an ever changing array of observations of the world. This is always true if the world is more complex than the perceptual system itself. It follows that the consistency of the matching between the $CS_I$ and the observed reality $CS_W$ can not be achieved within the system itself.
%Boyd \cite{} points out two kind of consistencies: 1. The consistency of the concept (conceptual structure) and 2 the consistency of the match up between the internal and the external conceptual structures. Both the conceptual structure and the mapping between conceptual structures are necessary incomplete.

Entropy represents the potential for doing work or taking action, high Entropy implies low potential for doing work or low capacity for taking action. %or high degree of confusion or disorder.
 Now the 2nd law of thermodynamics states that  Entropy must increase in a closed system. So whenever we try to do work or take action in a system, for example doing matching- we should anticipate an increase in Entropy. The point to grasp here is that any inward oriented effort to increase the matching within the system itself, will only increase the degree of mismatch, since it is irreversibly moving toward a higher, yet unknown state of confusion or Entropy.
 
The way out to beat the increasing disorder generated by the inward oriented system to talking to itself is to create higher and more general concepts of reality, or in IIT terms larger constellations with more higher order concepts.
This cycle repeats itself endlessly requiring at each time higher and broader levels of elaboration, that is larger and more structured  complexes. This unfolding drama is in reality only one part of a control mechanism that drives and regulates the alternating cycle. The goal-seeking effort is the other side of the control mechanism.
The increased in Entropy and the basic goal of increment existence in Matching seem to work as a dialectic engine. The result is a changing and expanding constellation of concepts (complexes) matched to a changing and expanding universe of observed reality.
 
The goal seeking effort to increase their capacity for ce power or existence when in a closed system, produces disorder and ultimately death, but on the other hand, when the the system is open to the world to match it, the increasing mismatch of the $CS_I$ with the observed reality $CS_W$ will open or unstructured the system. Thus, the same goal seeking effort, namely, maximize existence or cause-effect power (the E in the Matching formula), when the system is open as it is in matching, will work to reverse the march towards destruction or disorder.
Paradoxically then, the increment in entropy permits both the destruction or un-structuring of a closed sys and the creation of new system (complex) to nullify the march towards randomness.

\end{document}


\section{Rodrick Wallace}
Wallace proposes "introducing a
series of necessary conditions statistical models based on the asymptotic limit
theorems of information theory" to shorten the distance for applying Darwinism
to the social sciences.
To the 4 principles of Modern Evolutionary Synthesis (variation, heredity, mutation, reproduction), add a fifth -the principle of environmental interaction- individuals can exchange heritage material between markedly different organisms. there is no single privileged natural scale at which selection takes place.
%2.2
Predator pray has a grammar ABAB...Other statements are possible but we tend to observe the "grammar" ABABAB ...
The number of predators increases proportional to preys and the number of preys decreases proportional to predates , or dX/dt = wY, dY/dt = -wX.

More complex dynamical systems may have more complex dynamics (various attracts or grammars can be described with symbolic )
The goal is not taken symbolic dynamics as a simplification of more exact analytic or stochastic approaches  (physics-law principles) but identify grammars and syntaxes over the long term that are constrained by asymptotic probability theorems, leading to statistical methods similar to regression equations.
Suppose we can coarse-grain according to some appropriate partition of the phase-space.
Empirical identification of relevant coarse graining is not trivial , for example, if changes within the interval $\delta T$ becomes plastic or path dependent the attempts to model the system within as a Markov process within the natural interval $\delta T$ will fail.
Pattern is never a property of the system alone ore not the observer but of an interaction between them. Patterns exist at all levels and scales.
%2.4
(Atlan & Cohen)The essence of cognition i stye comparison of a perceived external signal (the external world) with an internal learned picture of the world, and the, upon comparison, the choice of one response upon a large repertoire of possible actions. Such reduction in  uncertainty inherently carries information. Thus there is convolution of an incoming external signal with an internal ongoing activity which included but is not limited to a a learned picture fp the world.
composite signals or paths $a_j= f(y_j,w_j)$, where y is the sensory input and w the ongoing internal activity. The composite path $x=a_0,a_1...$. We can define an ergodic (generalization of law of large numbers) information source X associated with stochastic variates $X_j$.

not all cognitive process are ergodic, this means that if the lim H = lim paths lead to action/all possible paths exists at all, is path dependent 




